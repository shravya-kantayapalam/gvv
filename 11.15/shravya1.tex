% \iffalse
\let\negmedspace\undefined
\let\negthickspace\undefined
\documentclass[journal,12pt,twocolumn]{IEEEtran}
\usepackage{cite}
\usepackage{amsmath,amssymb,amsfonts,amsthm}
\usepackage{algorithmic}
\usepackage{graphicx}
\usepackage{textcomp}
\usepackage{xcolor}
\usepackage{txfonts}
\usepackage{listings}
\usepackage{enumitem}
\usepackage{mathtools}
\usepackage{gensymb}
\usepackage{comment}
\usepackage[breaklinks=true]{hyperref}
\usepackage{tkz-euclide} 
\usepackage{listings}
\usepackage{gvv}                                        
\def\inputGnumericTable{}                                 
\usepackage[latin1]{inputenc}                                
\usepackage{color}                                            
\usepackage{array}                                            
\usepackage{longtable}                                       
\usepackage{calc}                                             
\usepackage{multirow}                                         
\usepackage{hhline}                                           
\usepackage{ifthen}                                           
\usepackage{lscape}

\newtheorem{theorem}{Theorem}[section]
\newtheorem{problem}{Problem}
\newtheorem{proposition}{Proposition}[section]
\newtheorem{lemma}{Lemma}[section]
\newtheorem{corollary}[theorem]{Corollary}
\newtheorem{example}{Example}[section]
\newtheorem{definition}[problem]{Definition}
\newcommand{\BEQA}{\begin{eqnarray}}
\newcommand{\EEQA}{\end{eqnarray}}
\newcommand{\define}{\stackrel{\triangle}{=}}
\theoremstyle{remark}
\newtheorem{rem}{Remark}
\begin{document}

\bibliographystyle{IEEEtran}
\vspace{3cm}

\title{11.15}
\author{EE23BTECH11030 - Shravya Kantayapalam}
\maketitle
\newpage
\bigskip

\renewcommand{\thefigure}{\theenumi}
\renewcommand{\thetable}{\theenumi}


\vspace{3cm}
\begin{document}
\textbf{Question:}\\
The transverse displacement of a string (clamped at its both ends) is given by
\begin{align*}
y(x, t) = 0.06 \sin{{2\pi/3}x} \cos{120 \pi t}
\end{align*}
where x and y are in m and t in s. The length of the string is 1.5 m and its mass is
$3.0 \times 10^{-2}$ kg.
Answer the following :
\begin{enumerate}
	\item[(a)] Does the function represent a travelling wave or a stationary wave?
	\item[(b)] Interpret the wave as a superposition of two waves travelling in opposite directions. What is the wavelength, frequency, and speed of each wave ?
	\item[(c)] Determine the tension in the string.\\
\end{enumerate}

\textbf{Solution:}\\
\begin{enumerate}
\begin{table}[h]
  \centering
  \begin{tabular}{|c|c|c|}
    \hline
    Variable & Description & Value \\
    \hline
    $\lambda$ & Wavelength & $\frac{1}{2\pi}$m\\
    \hline
    $v$ & Speed & $\frac{30}{\pi}$m/s\\
    \hline
    $f$ & Frequency & $60$Hz \\
    \hline
  \end{tabular}
  \caption{variables}
  \label{tab:mytable}
\end{table}
\item[(a)] The given function represents a stationary wave. This is evident from the presence of both sine and cosine terms. In a stationary wave, the displacement pattern does not propagate through space; instead, it oscillates in a fixed position.
\item[(b)] The given wave can be interpreted as a superposition of two waves traveling in opposite directions. This is because of the presence of both sine and cosine terms, which are characteristic of waves traveling to the right and left, respectively.
The general form of a wave moving to the right is:
\begin{align} 
y_1(x, t) = A \sin(kx - \omega t) 
\end{align} 
and the wave moving to the left is:
\begin{align} 
y_2(x, t) = A \sin(kx + \omega t) 
\end{align} 
In the given function 
\begin{align}
y(x, t) = 0.06 \sin(2 \pi x) \cos(120 \pi t)
\end{align}
we can express it as a superposition of two waves:

\begin{align}
y(x, t) = 0.06 \sbrak{\frac{1}{2} \sin(2\pi x - 120\pi t) + \frac{1}{2}\sin(2\pi x + 120\pi t)}
\end{align}

Comparing this with the general forms mentioned earlier, we can see that the waves are traveling in opposite directions. The wavelength $\lambda$ can be determined from the coefficient of  $x$ in the argument of the sine function. In this case,  $\lambda = \frac{1}{2\pi}$ meters. The frequency $f$ can be determined from the coefficient of $t$ in the argument of the cosine function. In this case, $f = \frac{120}{2} = 60$ Hz. The speed (\( v \)) of each wave can be found using the formula $v = f \lambda$. Subsituting the values, $v = 60 \times \frac{1}{2\pi}$ m/s.

\item[(c)] The speed of the wave $v$ can be related to the tension $T$ and mass per unit length  $\mu$ of the string through the equation $v = \sqrt{\frac{T}{\mu}}$. First, we need to find the mass per unit length  $\mu$. The mass  $m$ of the string is given as  $3.0 \times 10^{-2}$ kg, and the length $L$ is 1.5 m. So,  $\mu = \frac{m}{L}$. Substitute the values into the equation $v = \sqrt{\frac{T}{\mu}}$ and solve for tension  $T$ .
\end{enumerate}
\end{document}

