% \iffalse
\let\negmedspace\undefined
\let\negthickspace\undefined
\documentclass[journal,12pt,twocolumn]{IEEEtran}
\usepackage{cite}
\usepackage{amsmath,amssymb,amsfonts,amsthm}
\usepackage{algorithmic}
\usepackage{graphicx}
\usepackage{textcomp}
\usepackage{xcolor}
\usepackage{txfonts}
\usepackage{listings}
\usepackage{enumitem}
\usepackage{mathtools}
\usepackage{gensymb}
\usepackage{comment}
\usepackage[breaklinks=true]{hyperref}
\usepackage{tkz-euclide} 
\usepackage{listings}
\usepackage{gvv}                                        
\def\inputGnumericTable{}                                 
\usepackage[latin1]{inputenc}                                
\usepackage{color}                                            
\usepackage{array}                                            
\usepackage{longtable}                                       
\usepackage{calc}                                             
\usepackage{multirow}                                         
\usepackage{hhline}                                           
\usepackage{ifthen}                                           
\usepackage{lscape}

\newtheorem{theorem}{Theorem}[section]
\newtheorem{problem}{Problem}
\newtheorem{proposition}{Proposition}[section]
\newtheorem{lemma}{Lemma}[section]
\newtheorem{corollary}[theorem]{Corollary}
\newtheorem{example}{Example}[section]
\newtheorem{definition}[problem]{Definition}
\newcommand{\BEQA}{\begin{eqnarray}}
\newcommand{\EEQA}{\end{eqnarray}}
\newcommand{\define}{\stackrel{\triangle}{=}}
\theoremstyle{remark}
\newtheorem{rem}{Remark}
\begin{document}


\title{11.15}
\author{EE23BTECH11030 - Shravya Kantayapalam}
\maketitle
\newpage
\bigskip

\renewcommand{\thefigure}{\theenumi}
\renewcommand{\thetable}{\theenumi}

\textbf{Question:}\\
The transverse displacement of a string (clamped at its both ends) is given by
\begin{align}
y\brak{x, t} = 0.06 \sin{\brak{\dfrac{2\pi}{3}x}} \cos{\brak{120 \pi t}}
\end{align}
where x and y are in m and t in s. The length of the string is 1.5 m and its mass is
$3.0 \times 10^{-2}$ kg.
Answer the following :
\begin{enumerate}
	\item[(a)] Does the function represent a travelling wave or a stationary wave?
	\item[(b)] Interpret the wave as a superposition of two waves travelling in opposite directions. What is the wavelength, frequency, and speed of each wave ?
	\item[(c)] Determine the tension in the string.\\
\end{enumerate}

\solution \\
\begin{table}[h]
  \centering
\begin{tabular}{|m{10em}|m{10em}|}
\hline
\textbf{Standing Wave} & \textbf{Traveling Wave} \\
\hline
Formed by the interference of two waves traveling in opposite directions & Propagates through a medium without interference \\
\hline
Pattern appears stationary, with nodes and antinodes & Waveform moves through space \\
\hline
No net transport of energy & Transports energy from one place to another \\
\hline
Nodes do not move, so velocity is zero & Moves with a constant velocity \\
\hline
\end{tabular}
 \caption{Table-1:Difference between standing and travelling wave}
 \label{tab:my}
\end{table}
\begin{enumerate}
\item[(a)] The given function represents a stationary wave. In a stationary wave, the displacement pattern does not propagate through space, instead, it oscillates in a fixed position.The given wave can be interpreted as a superposition of two waves traveling in opposite directions as explained in next part.
\item[(b)] $y\brak{x,t}$ can be represented as
\begin{align} 
y_1\brak{x, t} &= -A \sin{\brak{ \omega t - kx}}\\ 
y_2\brak{x, t} &= A \sin{\brak{\omega t + kx}} \\
y\brak{x,t} &= y_1\brak{x,t} + y_2\brak{x,t}\\
&= 2A\sin{\brak{kx}}\cos{\brak{\omega t}}
\label{eq:1}
\end{align} 
In the given function 
\begin{align}
y\brak{x, t} = 0.06 \sin{\brak{\dfrac{2\pi}{3}x}} \cos{\brak{120 \pi t}} \label{eq:2}
\end{align}
Camparing \eqref{eq:1} and \eqref{eq:2} then
\begin{align} 
y_1\brak{x, t} &= -0.03 \sin{\brak{120\pi t - \dfrac{2\pi}{3}x}}\\ 
y_2\brak{x, t} &= 0.03 \sin{\brak{ 120\pi t + \dfrac{2\pi}{3}x}}
\end{align}
we know
\begin{align}
\lambda &= \dfrac{2\pi}{k}\\
f &= \dfrac{\omega}{2\pi}\\
v &= \lambda f
\end{align}
\begin{table}[h]
  \centering
  \begin{tabular}{|c|c|c|}
    \hline
    Variable & Description & Value \\
    \hline
    $\lambda$ & Wavelength & $3$m\\
    \hline
    $f$ & Frequency & $60$Hz \\
    \hline
    $v$ & Speed & $180$m/s\\
    \hline
  \end{tabular}
  \caption{Table-2:wavelength, frequency and velocity of $y_1(x,t)$}
  \label{tab:mytable1}
\end{table}
\begin{table}[h]
  \centering
  \begin{tabular}{|c|c|c|}
    \hline
    Variable & Description & Value \\
    \hline
    $\lambda$ & Wavelength & $3$m\\
    \hline
    $f$ & Frequency & $60$Hz \\
    \hline
    $v$ & Speed & $180$m/s\\
    \hline
  \end{tabular}
  \caption{Table-3:wavelength, frequency and velocity of $y_2(x,t)$}
  \label{tab:mytable2}
\end{table}
\item[(c)] Velocity in terms of tension can be written as
\begin{align}
v&=\sqrt{\dfrac{T}{\mu}}\\
T&=v^2\mu
\end{align}
$\mu$ is mass per unit length
\begin{align}
\mu&=\dfrac{\brak{\dfrac{3}{100}}}{1.5}\\
&=\dfrac{1}{50}\\
T&=v^2\mu\\
&=\brak{180}^2\brak{\dfrac{1}{50}}\\
&=648
\end{align}
Tension in string = 648 N
\end{enumerate}
\end{document}

